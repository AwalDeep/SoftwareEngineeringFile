\chapter{Conduct feasibility study for Text Classifier.}
\label{Practical:1}

\section{Feasibility study}
A feasibility study is a study, which is performed by an organization in order to evaluate whether a specific action makes sense from an economic or operational standpoint. The objective of the study is to test the feasibility of a specific action and to determine and define any issues that would argue against this action.\\
The question a feasibility study essentially tries to answer is:“Should we proceed with the specific action plan?” On top of determining whether the plan is viable/, organizations can use a feasibility study for understanding the risks better and preparing for them.
\subsection{When should a feasibility study be used}
While feasibility studies are typically conducted by business organizations, other organizations can naturally benefit from it as well. Since the study aims to discover whether an action is viable, it can help organizations to avoid costly or operationally exhausting ventures.\\
The study is typically used in situations where an important strategic decision needs to be taken.
\subsection{Core elements of feasibility study}
\subsubsection{Technical feasibility}
The first element deals with technical feasibility of the proposed action plan. If your organization is introducing a new product or a service, the technical feasibility study will determine if it’s a technically viable action.\\
Classifier is a service which enables users to get feedbacks of their posts and products. Because of unique nature it has very less competition. You can get feedback reports quite easily without doing much. \\
Nowadays every business wants to excel that’s why this software can be useful for them. Other than that no special requirements are needed

\subsubsection{Market feasibility}
The second element focuses on testing the market for the proposed action or idea. It examines issues like whether the product or service can be sold at reasonable prices or if there’s a marketplace for it. Basically the software is for everyone who want to review their products and posts. But majorly it is targeting users in social media.\\
There are a lot of similar applications but still no one has completely solved the problem, so this will attract users.

\subsubsection{Commercial feasibility}
Commercial feasibility is an element of the study focused on the probability of commercial success. It’s mainly focused on studying the new business or a new product or service, and whether your organization can create enough profit with it.
\\
The app means helpful to all in many purposes. It can survive without any activity for a short period of time. The cost is usually default values of data, maintaining of application etc. The strength of app is that it’s very unique. The privacy of user is not being violated at all unless the user wants to reveal his identity. Although, at higher level, external finance is much needed for its better working.

\subsubsection{Overall risk assessment}
The fourth element focuses on the major risks the proposed plan can entail. The overall risk assessment part of a feasibility study examines the different ways your organization can reduce the risk of embarking on the new action.\\
The profits in this software is that with the help of this user can review its contents which means everything to the developers.

\subsubsection{Operational feasibility}
Operational feasibility is the measure of how well a proposed system solves the problems, and takes advantage of the opportunities identified during scope definition and how it satisfies the requirements identified in the requirements analysis phase of system development.\\
If it accurately identifies the posts as negative, positive or neutral  for majority of the queries than we can say that it is operationally feasible. 

\subsubsection{Schedule Feasibility}
A project will fail if it takes too long to be completed before it is useful. Typically this means estimating how long the system will take to develop, and if it can be completed in a given time period using some methods like payback period. Schedule feasibility is a measure of how reasonable the project timetable is. Some projects are initiated with specific deadlines. It is necessary to determine whether the deadlines are mandatory or desirable.\\
Given our technical expertise, the project deadlines are reasonable and thus have a feasible schedule.

