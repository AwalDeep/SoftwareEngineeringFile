\chapter{Case Study of OpenProj}
\label{Practical:4}
\section{History and current status}
OpenProj was developed at Projity by Marc O'Brien, Howard Katz and Laurent Chretienneau
in 2007. It moved out of beta with the release of Version 1.0, on January 10, 2008.\\
In late 2008, Projity was acquired by Serena Software. As of early 2009 support for OpenProj
and communication about development of OpenProj seem to have been suspended. There were
actually regressions with a few commits to the CVS, There has been no improvement in the
past four years, and it is no longer compatible with Microsoft Project.\\
In 2012, the founders of OpenProj forked the abandoned code base of OpenProj and started
development for a new release. The initial release of this fork occurred in August 2012. The
name of the new fork is ProjectLibre
\section{Features}
The current version includes:
\begin{itemize}
	\item Earned value costing.
	\item Gantt chart.
	\item PERT chart.
	\item Resource Breakdown Structure (RBS) chart. 
	\item Task usage reports.
	\item Work Breakdown Structures (WBS) chart.
	
\end{itemize}
\section{Popularity}
It has been downloaded over 4,000,000 times in over 142 countries. Three months after the
beta version release, on SourceForge an average of 60,000 copies a month were downloaded.
With a SourceForge activity percentile of 99.964, at number 15 it was listed just ahead of the
popular messaging application Pidgin. In May 2008 the total number of downloads on
SourceForge reached 500,000.

\section{Bugs}
As of version 1.4, bugs in the software generally only manifest for users who are attempting
more advanced features. For example, tasks may mysteriously start at a certain time (they
behave as if they have a 'Start no earlier than' constraint even though none exists, and the project
start date is not a constraint), links show gaps, fixed cost for summary tasks neither sums nor
is editable, etc. Sometimes these errors are solved by restarting the software, but others are
persistent. Compared to Microsoft Project, which it closely emulates, OpenProj has a similar
user interface (UI), and a similar approach to construction of a project plan: create an indented
task list or work breakdown structure (WBS), set durations, create links (either by (a) mouse
drag, (b) selection and then button-down, or (c) manually type in the 'predecessor' column),
assign resources. The columns (fields) are the same as for Microsoft Project. Users of either
software should be broadly comfortable using the other. Costs are the same: labour, hourly rate,
material usage, and fixed costs: these are all provided.\\
However, there are small differences in the UI (comments apply to version 1.4), which take
some adaptation for those familiar with Microsoft Project, i.e. OpenProj can't link upwards
with method (c), inserting tasks is more difficult than in Microsoft Project, and OpenProj can't
create resources on the fly (have to create them first in the resource sheet). There are also
several more serious limitations with OpenProj, the chief of these being the unavailability of
more detailed views and reports typical of Microsoft Project. For example, though the fields
exist for cost, there is no quick way to show them other than to manually insert them. This
requires a relatively advanced user: someone who knows what the fields might be called and
how to use them.
\section{Licensing}
Some features of OpenProj are limited to users acquiring a purchased license; for those users
using OpenProj for free, a slightly limited feature set is provided. For example, OpenProj(v1.4)
does not allow the in-house exporting of PDF output, though the usefulness of such a feature
is questionable. It is possible to circumvent the reduced feature set using external software,
though as with all paid software, donation or purchase is appreciated by the developers.

\section{Derivatives}
ProjectLibre: The original founders of OpenProj started to develop a complementary server
for OpenProj in 2012, comparable to Microsoft Project Server for Microsoft Project. During
development they realized, that the fact that OpenProj had not been updated anymore by Serena
Software during the last four years will become problematic to their goal, so they needed to
develop first a significantly updated version of OpenProj. This version was released as a fork
called PrjectLibre in August 2012. The complementary server will be called ProjectLibre
Server. ProjectLibre corrects many issues of OpenProj and introduces significant features such
as:
\begin{itemize}
	\item Import/export with Microsoft Project 2010.
	\item Printing.
	\item PDF exporting (without any restrictions)
	\item New ribbon user interface.
	\item Full compatibility with Microsoft Project 2010.
	\item Many bug fixes and correction of issues that OpenProj encounters.
\end{itemize}