\chapter{Preparation of software configuration and risk management related documents}
\label{Practical:4}
\section{Software configuration}
The results of a large software development effort typically consist of large number of objects e.g. source code,
design document, SRS document, test document, user’s manual etc. These objects are usually referred to and
modified by a number of software developers throughout the lifecycle of the software. The state of all these
objects at any point of time is called the configuration of the software product. The state of each deliverable
object changes as development progresses and also as bugs are detected and fixed. Software configuration
management deals with effectively tracking and controlling the configuration of a software product during its
lifecycle. 

\subsection{Configuration management activities } 
A configuration management tool provides automated support for overcoming all the problems. In addition, a
configuration management tool helps to keep track of various deliverable objects, so that the project manager
can quickly and unambiguously determine the current state of project. The configuration management tool
enables the developer to change various components in a controlled manner.
Configuration management is carried out through principal activities: 
\begin{itemize}
\item Configuration identification involves deciding which parts of the system should be kept track of. We
can identify the objects into three categories controlled, pre controlled and uncontrolled
\item Configuration control ensures that changes to a system happen smoothly. If we make any changes to
the input systems then we will get the outputs as per the inputs. If we make changes in one module like
user login then same changes must be done smoothly in the user register module.\\ If we add some feature
in our software then it must be updated in user part also.  
 	
\end{itemize}	
\section{Risk management}
\subsection{Risk identification}
The project manager needs to anticipate the risks in the project as early as possible so that the impact of the
risks can be minimized by making effective risk management plans. A project can be affected by a large variety
of risks which can led down the company badly. There are three main categories of risks which can affect a
software project as follows:
\begin{itemize}
\item Project risks concern various form budgetary, schedule, personnel, resource and customer related
problems. In this application it is risk, if the details of one person gets reveal to the other person.\\
The use of sentiment analysis and related monitoring technology may not be a solution for everyone. “Today it’s understood that any information that passes through a corporation—in an email, phone conversation or chat session—belongs to the corporation and can be used in a regulatory context,” said Seth Grimes, an analytics consultant.
\item Technical risk concern potential design, implementation, interfacing, testing and maintenance
problems. The risk can be if data does not entered into the database or wrong entry gets entered into the database 
table or one table entry gets entered into the other table.
\\
There are some technical risks involved with sentiment analysis.\\
Even in the best of circumstances, it is only 65 to 70 percent accurate, said Susan Etlinger, an analyst with research firm Altimeter Group. She noted that the accuracy rate drops even further when the process is applied to text in languages other than English.
\item Business risks : These type of risks include risks of building an excellent product that no one wants, losing
budgetary or personnel commitments.\\
Companies have to decide whether or not they are comfortable with this level of monitoring done in sentiment analysis. Could there be an adverse reaction from employees if they know they are being monitored? Will it present a challenge to recruitment? 

\end{itemize}
\subsection{Risk Assessment }
The objective of risk assessment is to rank the risks in terms of their damage causing potential. For risk
management, first each risk should be rated in two ways: 
\begin{itemize}
\item The likelihood of a risk coming true (r). 
\item The consequence of the problems associated with that risk (s).	
\end{itemize}
Based on these two factors, the priority of each risk completed: \\
p=r*s \\
where , p is the priority with which the risk must be handled, r is the probability of the risk becoming true,
and s is the severity of damage caused due to risk becoming true. If all identified risks are prioritized, then the
most likely and damaging risks can be handled first and more comprehensive risk abatement procedures can
be designed for these risks. 
\subsection{Risk Containment}
After all the identified risks of a project are assessed, plans must be made to first contain the most damaging
and the most likely risks. Different risks require different containment procedures. In fact, most risks require
ingenuity on the part of the project manager in tackling the risks. There are three main strategies to plan for
risk containment:
\begin{itemize}
\item Avoid the risk: In some cases, you may want to avoid the risk altogether. This could mean not getting
involved in a business venture, passing on a project, or skipping a high-risk activity. This is a good
option when taking the risk involves no advantage to your organization, or when the cost of addressing
the effects in not worthwhile.
\item  Share the risk: You could also opt to share the risk- and the potential gain- with other people, teams,
organizations or third parties. For instance, you share the risk when you insure your office building and
your inventory with a third party insurance company, or when you partner with another organization in
a joint product development initiative.
\item  Accept the risk: Your last option is to accept the risk. This option is usually best when there’s nothing
you can do to prevent or mitigate a risk, when the potential loss is less than the cost of insuring against
the risk, or when the potential gain is worth accepting the 